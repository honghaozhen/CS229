\item \subquestionpoints{5} \textbf{Fisher Information (alternate form)}

It turns out that the Fisher information can not only be defined as the covariance of the score function,
but in most situations it can also be represented as the expected negative Hessian of the log-likelihood.

Show that $\E_{y\sim p(y;\theta)}[-\nabla^2_{\theta'} \log p(y;\theta')|_{\theta'=\theta}]=\mathcal{I}(\theta)$.

\textbf{Remark}. The Hessian represents the curvature of a function at a point. This shows that the expected curvature of the log-likelihood function is also equal to the Fisher information matrix. If the curvature of the log-likelihood at a parameter is very steep (i.e. Fisher information is very high), this generally means you need fewer data samples to a estimate that parameter well (assuming data was generated from the distribution with those parameters), and vice versa. The Fisher information matrix associated with a statistical model parameterized by $\theta$ is extremely important in determining how a model behaves as a function of the number of training set examples.

